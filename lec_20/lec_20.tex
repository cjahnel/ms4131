\documentclass[a4paper, 12pt]{article}
\usepackage{amsmath, amssymb}

\newtheorem{theorem}{Theorem}[section]
\newtheorem{lemma}[theorem]{Lemma}
\newtheorem{proposition}[theorem]{Proposition}
\newtheorem{corollary}[theorem]{Corollary}

\newenvironment{proof}[1][Proof]{\begin{trivlist}
\item[\hskip \labelsep {\bfseries #1}]}{\end{trivlist}}
\newenvironment{definition}[1][Definition]{\begin{trivlist}
\item[\hskip \labelsep {\bfseries #1}]}{\end{trivlist}}
\newenvironment{example}[1][Example]{\begin{trivlist}
\item[\hskip \labelsep {\bfseries #1}]}{\end{trivlist}}
\newenvironment{remark}[1][Remark]{\begin{trivlist}
\item[\hskip \labelsep {\bfseries #1}]}{\end{trivlist}}

\newcommand{\qed}{\nobreak \ifvmode \relax \else
    \ifdim\lastskip<1.5em \hskip-\lastskip
    \hskip1.5em plus0em minus0.5em \fi \nobreak
    \vrule height0.75em width0.5em depth0.25em\fi}
\newcommand{\keyword}[1]{\textbf{#1}}

\title{MS4131 Lecture 20: Diagonalization}
\author{Kevin Moroney}
\date{30 November 2023}

\begin{document}
    \maketitle
    \section{Diagonalization}
    \begin{definition}
        A square matrix $A$ is said to be \keyword{diagonalizable} if there exists an
        invertible matrix $P$ such that $P^{-1}AP$ is diagonal.
    \end{definition}
    \begin{theorem}
        If $A$ is an $n \times n$ matrix, then the following are equivalent:
        \begin{enumerate}
            \item $A$ is diagonalizable
            \item $A$ has $n$ linearly independent eigenvectors
        \end{enumerate}
    \end{theorem}
    \begin{remark}
        If $A$ does not have $n$ linearly independent eigenvectors, it is not
        diagonalizable.
    \end{remark}
    \subsection*{Application}
    $A^n = (PDP^{-1})^n \because P^{-1}AP = D \implies A = PDP^{-1} \therefore A^n = PD^nP^{-1}$
    \section*{Orthogonal diagonalization}
    \begin{definition}
        A square matrix $A$ with the property $A^{-1} = A^T$ is said to be an
        \keyword{orthogonal matrix}.
    \end{definition}
    \begin{definition}
        A square matrix $A$ is \keyword{orthogonally diagonalizable} if there is an
        orthogonal matrix $P$ such that $P^{-1}AP = P^TAP$ is diagonal.
        The matrix $P$ is said to orthogonally diagonalize $A$.
    \end{definition}
    \begin{remark}
        If $A$ is orthogonally diagonalizable then $P^{-1}AP$ is diagonal
        where $P$ is orthogonal;\\
        that is $P^TP = I$ or if $p_i$ for $i = 1, 2, \dots, n$ are the columns of $P$
        then
        \[
            p_i^Tp_j = \delta_{ij} =
            \begin{cases}
                1 \quad \text{ if } i = j \\
                0 \quad \text{ if } i \ne j \\
            \end{cases}
        \]
        that is, the columns of $P$ (the eigenvectors of $A$) are
        \keyword{orthonormal} and so $A$ has $n$ orthonormal eigenvectors.\\
        Similarly if $A$ has $n$ orthonormal eigenvectors $p_1, p_2, \dots, p_n$ then
        the matrix $P$, whose columns are these eigenvectors,
        diagonalizes $A$ orthogonally.\\
        Thus $A$ is orthogonally diagonalizable if and only if $A$ has $n$
        orthonormal eigenvectors.
    \end{remark}
    \begin{theorem}
        If $A$ is an $n \times n$ matrix, then $A$ is orthogonally diagonalizable if
        and only if $A$ is symmetric.
    \end{theorem}
    \begin{theorem}
        If $A$ is symmetric, then eigenvectors corresponding to different
        eigenvalues are orthogonal.
    \end{theorem}
\end{document}