\documentclass[a4paper, 12pt]{article}
\usepackage{amsmath, amssymb}

\newtheorem{theorem}{Theorem}[section]
\newtheorem{lemma}[theorem]{Lemma}
\newtheorem{proposition}[theorem]{Proposition}
\newtheorem{corollary}[theorem]{Corollary}

\newenvironment{proof}[1][Proof]{\begin{trivlist}
\item[\hskip \labelsep {\bfseries #1}]}{\end{trivlist}}
\newenvironment{definition}[1][Definition]{\begin{trivlist}
\item[\hskip \labelsep {\bfseries #1}]}{\end{trivlist}}
\newenvironment{example}[1][Example]{\begin{trivlist}
\item[\hskip \labelsep {\bfseries #1}]}{\end{trivlist}}
\newenvironment{remark}[1][Remark]{\begin{trivlist}
\item[\hskip \labelsep {\bfseries #1}]}{\end{trivlist}}

\newcommand{\qed}{\nobreak \ifvmode \relax \else
      \ifdim\lastskip<1.5em \hskip-\lastskip
      \hskip1.5em plus0em minus0.5em \fi \nobreak
      \vrule height0.75em width0.5em depth0.25em\fi}
\newcommand{\keyword}[1]{\textbf{#1}}

\title{MS4131 Lecture 17: Planes and Lines (in $\mathbb{R}^2$ and $\mathbb{R}^3$)}
\author{Kevin Moroney}
\date{20 November 2023}

\begin{document}
    \maketitle
    \section{Lines in $\mathbb{R}^2$}
        The general equation of a straight line in $\mathbb{R}^2$ is
        \begin{equation}
            ax + by + c = 0.
        \end{equation}
        \begin{theorem}
            The vector $n = (a, b)$ is orthogonal to the line given by equation (1).
        \end{theorem}
        \subsection*{Distance between a point and a line in $\mathbb{R}^2$}
        \begin{theorem}
            The \keyword{distance} $d$ between a point $P_0 = (x_0, y_0)$ and the line
            $l: ax + by + c = 0$ is
            \[
                d = \frac{|ax_0 + by_0 + c|}{\sqrt{a^2 + b^2}}.
            \]
        \end{theorem}
    \section{Planes in $\mathbb{R}^3$}
        A \keyword{plane} in $\mathbb{R}^3$ is determined by one of its
        points and its slope or inclination, which is given by a \keyword{normal} to the plane.
        \begin{definition}
            A plane $\pi$ through the point $P_0 = (x_0, y_0, z_0)$, with normal
            $n = (a, b, c)$, is the set of points $P = (x, y, z)$ such that
            \[
                n \cdot \overrightarrow{P_0P} = 0.
            \]
        \end{definition}
        Therefore the equation of $\pi$ is given by
        \[
            a(x - x_0) + b(y - y_0) + c(z - z_0) = 0.
        \]
        This equation is called the \keyword{point-normal form} of the equation of
        the plane.\\
        This can be rewritten as
        \begin{equation*}
            (a, b, c) \cdot (x - x_0, y - y_0, z - z_0) = 0,
        \end{equation*}
        which is called the \keyword{vector form} of the equation of the plane.\\
        And in turn can also be rewritten as
        \begin{equation*}
            ax + by + cz + d = 0,
        \end{equation*}
        called the \keyword{general form} of the equation of the plane.
        \begin{theorem}
            If $a, b, c, d \in \mathbb{R}$, with $a, b, c$ not all zero, then
            \begin{equation*}
                ax + by + cz + d = 0
            \end{equation*}
            is the equation of a plane with $n = (a, b, c)$ as a normal.
        \end{theorem}
        \begin{remark}
            Solving the system
            \[ax + by = k_1\]
            \[cx + dy = k_2\]
            means finding the point(s) $P = (x, y)$ of intersection of two lines
            in $\mathbb{R}^2$.\\
            Solving the system
            \[a_{11}x + a_{12}y + a_{13}z = k_1\]
            \[a_{21}x + a_{22}y + a_{23}z = k_1\]
            \[a_{31}x + a_{32}y + a_{33}z = k_1\]
            means finding the point(s) $P = (x, y, z)$ of intersection of three
            planes in $\mathbb{R}^3$.
        \end{remark}
    \section{Lines in $\mathbb{R}^3$}
        In $\mathbb{R}^3$, a line is determined by a point and a direction:
        the direction is simply given by a \keyword{nonzero vector}.\\
        Let $l$ denote the line passing through the point $P_0 = (x_0, y_0, z_0)$
        and parallel to the nonzero vector $v = (a, b, c)$ (or equivalently
        with direction $v = (a, b, c)$).\\
        Then $l$ is the set of points $P = (x, y, z)$ such that
        \[ \overrightarrow{P_0P} \text{ is parallel to } v. \]
        i.e. $l$ is the set of points $P = (x, y, z)$ such that
        \[ \overrightarrow{P_0P} = tv, \text{ for some } t. \]
        In terms of components, we have
        \begin{equation*}
            (x - x_0, y- y_0, z - z_0) = (ta, tb, tc), \quad \text{ for some }t.
        \end{equation*}
        Therefore, the equation of the line $l$ is
        \[
            \begin{cases}
                x = x_0 + ta \\
                y = y_0 + tb \\
                z = z_0 + tc
            \end{cases}
            , \quad
            \text{for all $-\infty < t < +\infty$.}
        \]
        If $t$ varies between $-\infty$ and $+\infty$, then the point $P = (x, y, z)
        = (x_0 + ta, y_0 + tb, z_0 + tc)$ describes the entire line $l$.
        \begin{definition}
            Equations
            \[
                \begin{cases}
                    x = x_0 + ta \\
                    y = y_0 + tb \\
                    z = z_0 + tc
                \end{cases}
                , \quad
                \text{for all $-\infty < t < +\infty$.}
            \]
            are called the \keyword{parametric equations} for the line $l$.
        \end{definition}
        \subsection*{Distance between a point and a plane in $\mathbb{R}^3$}
            \begin{theorem}
                The \keyword{distance $D$} between a point $P_0 = (x_0, y_0, z_0)$ and a plane
                $\pi$: $ax + by _ cz + d = 0$ is
                \begin{equation*}
                    D = \frac{|ax_0 + by_0 + cz_0 + d|}{\sqrt{a^2 + b^2 + c^2}}.
                \end{equation*}
            \end{theorem}
        \subsection*{Distance between two parallel planes}
            If $\pi$ and $\pi'$ are two parallel planes, then if we choose a point
            $P_0 \in \pi$ for example, the distance between $\pi$ and $\pi'$ is equal to
            the distance between $P_0$ and $\pi'$.
\end{document}
