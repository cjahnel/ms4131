\documentclass[a4paper, 12pt]{article}
\usepackage{amsmath, amssymb}

\newtheorem{theorem}{Theorem}[section]
\newtheorem{lemma}[theorem]{Lemma}
\newtheorem{proposition}[theorem]{Proposition}
\newtheorem{corollary}[theorem]{Corollary}

\newenvironment{proof}[1][Proof]{\begin{trivlist}
\item[\hskip \labelsep {\bfseries #1}]}{\end{trivlist}}
\newenvironment{definition}[1][Definition]{\begin{trivlist}
\item[\hskip \labelsep {\bfseries #1}]}{\end{trivlist}}
\newenvironment{example}[1][Example]{\begin{trivlist}
\item[\hskip \labelsep {\bfseries #1}]}{\end{trivlist}}
\newenvironment{remark}[1][Remark]{\begin{trivlist}
\item[\hskip \labelsep {\bfseries #1}]}{\end{trivlist}}

\newcommand{\qed}{\nobreak \ifvmode \relax \else
      \ifdim\lastskip<1.5em \hskip-\lastskip
      \hskip1.5em plus0em minus0.5em \fi \nobreak
      \vrule height0.75em width0.5em depth0.25em\fi}
\newcommand{\keyword}[1]{\textbf{#1}}

\title{MS4131 Lecture 19: Complex Eigenvalues and Linear Independence}
\author{Kevin Moroney}
\date{27 November 2023}

\begin{document}
    \maketitle
    \section*{Complex eigenvalues}
    \subsection*{Complex numbers}
    A \keyword{complex number} $z \in \mathbb{C}$ takes the form
    \[z = x + iy\]
    where $x, y \in \mathbb{R}$ and $i = \sqrt{-1}$.\\
    The \keyword{complex conjugate} of $z$ is $\bar{z} = x - iy$.\\
    \subsection*{Modulus}
    \[z\bar{z} = (x + iy)(x - iy) = x^2 + ixy - ixy + y^2 = x^2 + y^2\]
    The \keyword{modulus} of $z$ is
    \[r = |z| = \sqrt{z\bar{z}} = \sqrt{x^2 + y^2}\]
    If the \keyword{imaginary part} of $z$ is zero; i.e., if $y = 0$:
    \[|z| = \sqrt{x^2} = |x|\]
    then the modulus is the \keyword{absolute value} of $x$ (where $x$ is the
    \keyword{real part} of $z$).
    \subsection*{Complex vectors and matrices}
    Similarly a complex vector $u \in \mathbb{C}^n$ can be written as $u = v + iw$,
    where $v$ and $w$ are real valued vectors.\\
    The norm of the complex vector
    \[
        u = \begin{pmatrix}
                u_1\\
                u_2\\
                \vdots\\
                u_n
            \end{pmatrix}
    \]
    is
    \[
        ||u|| = \sqrt{\sum_{i=1}^{n}u_i\bar{u_i}}
        = \sqrt{u_1\bar{u_1} + u_2\bar{u_2} + \dots + u_n\bar{u_n}}
    \]
    \begin{remark}
        If $\lambda$ is a complex eigenvalue of a real matrix $A$ then its
        conjugate $\bar{\lambda}$ is also an eigenvalue.
    \end{remark}
    \begin{theorem}
        If $A \in \mathbb{R}^{n \times n}$ is symmetric then all its eigenvalues are \keyword{real}.
    \end{theorem}
    \section*{Linear independence}
    \begin{definition}
        If $S = \{v_1, v_2, ..., v_r\}$ is a set of vectors, then the equation
        \[k_1v_1 + k_2v_2 + ... + k_rv_r = 0\]
        has \keyword{at least one solution}: $k_i = 0 \; \forall i$.\\
        If there is no other solution, then $S$ is a linearly independent
        set. Otherwise $S$ is linearly dependent.
    \end{definition}
    \begin{remark}
        If $S$ is linearly dependent, then at least one of the scalars $\{k_i\}$ is
        non-zero. Assume $k_j \neq 0$; then
        \[v_j = -\frac{k_1}{k_j}v_1 - \frac{k_2}{k_j}v_2 - \dots - \frac{k_r}{k_j}v_r\]
        i.e. $v_j$ is a linear combination of the other vectors in the set.
    \end{remark}
    \begin{remark}
        2 vectors in $\mathbb{R}^2$ or $\mathbb{R}^3$ are lienarly dependent if and only if they
        lie on the same line through the origin and 3 vectors in $\mathbb{R}^3$ are
        linearly dependent if and only if they lie on the same plane
        through the origin.
    \end{remark}
    \subsection*{Matrix with linearly independent columns}
    Let $X$ be the matrix $(x_1, x_2, \dots, x_n)$ whose columns $x_1, x_2, \dots, x_n$
    are linearly independent.\\
    Let $k = \begin{pmatrix}
                k_1\\
                k_2\\
                \vdots\\
                k_n
        \end{pmatrix}$. Then
    \[X^Tk = k_1x_1 + k_2x_2 + \dots + k_nx_n = 0 \iff k = 0\]
    Thus $X^T$ is invertible and so $X$ is invertible.
\end{document}